\documentclass[tikz,border=10pt]{standalone}
\usepackage{tikz}
\usepackage{ctex} % 支持中文
\usetikzlibrary{arrows.meta,shapes,positioning,calc,fit}

\begin{document}
\begin{tikzpicture}[
    node distance=1cm,
    box/.style={rectangle, draw, rounded corners, minimum width=4cm, minimum height=3cm, fill=#1, text width=3.8cm, align=center},
    tsp-node/.style={circle, draw, fill=cyan!30, minimum size=0.8cm, font=\bfseries},
    edge/.style={->, >=stealth, thick},
    weight/.style={font=\small, fill=white, inner sep=1pt, circle},
    title/.style={font=\bfseries},
    description/.style={fill=white, rounded corners, inner sep=5pt},
    arrow/.style={-{Stealth[scale=1.5]}, thick, red},
    list-item/.style={font=\small}
]

% 标题
\node[title, font=\Large] at (0,5) {TSP与SAT问题的关系};

% TSP示例框
\node[box=blue!10] (tsp) at (-4,1) {};
\node[title] at ($(tsp.north)+(0,-0.3)$) {TSP问题示例};

% TSP图
\node[tsp-node] (A) at ($(tsp.center)+(-1.5,-0.5)$) {A};
\node[tsp-node] (B) at ($(tsp.center)+(-0.75,0.5)$) {B};
\node[tsp-node] (C) at ($(tsp.center)+(0,0)$) {C};
\node[tsp-node] (D) at ($(tsp.center)+(0.75,0.5)$) {D};
\node[tsp-node] (E) at ($(tsp.center)+(1.5,-0.5)$) {E};

% TSP边
\draw[edge] (A) -- (B) node[weight, midway, above left] {5};
\draw[edge] (B) -- (C) node[weight, midway, above] {4};
\draw[edge] (C) -- (D) node[weight, midway, above] {6};
\draw[edge] (D) -- (E) node[weight, midway, above right] {3};
\draw[edge, dashed] (E) -- (A) node[weight, midway, below] {8};
\draw[edge] (A) -- (C) node[weight, midway, below left] {7};

% TSP描述
\node[description] at ($(tsp.south)+(0,0.5)$) {目标:找出访问所有城市\\并返回起点的最短路径};

% SAT示例框
\node[box=green!10] (sat) at (4,1) {};
\node[title] at ($(sat.north)+(0,-0.3)$) {SAT问题示例};

% SAT公式
\node[description] (formula) at ($(sat.center)+(0,0)$) {
\begin{tabular}{l}
(x1 OR x2 OR NOT x3) AND\\
(NOT x1 OR NOT x2 OR x4) AND\\
(x2 OR NOT x3 OR NOT x4) AND\\
(x1 OR x3 OR x4)
\end{tabular}
};

% SAT描述
\node[description] at ($(sat.south)+(0,0.5)$) {目标:找出使公式为真的变量赋值};

% 归约箭头
\draw[arrow] ($(tsp.east)+(0,0)$) -- ($(sat.west)+(0,0)$);
\node[title, red] at (0,1.5) {多项式时间归约};

% 变量设计
\node[draw, rounded corners, fill=white, text width=5cm, align=left] (vars) at (0,-1) {
\textbf{变量设计:}
\begin{itemize}[leftmargin=*]
\item[*] x[i,j,p]: 城市i在位置p访问城市j
\item[*] 位置唯一性约束
\item[*] 访问唯一性约束
\item[*] 路径连续性约束
\item[*] 路径长度约束
\end{itemize}
};

% 理论意义
\node[box=orange!20] (theory) at (0,-4.5) {};
\node[title] at ($(theory.north)+(0,-0.3)$) {理论意义};

\node[text width=5cm, align=left] at ($(theory.center)+(0,-0.3)$) {
\begin{itemize}[leftmargin=*]
\item[*] 证明了TSP和SAT都是NP完全问题
\item[*] 归约建立了问题之间的桥梁
\item[*] 任何一个问题的多项式解法都意味着P=NP
\item[*] 为算法设计提供了不同视角
\item[*] 启发了混合求解策略的开发
\end{itemize}
};

\end{tikzpicture}
\end{document} 